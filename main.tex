\documentclass[a4paper,11pt]{article}

% General  instructions / Best Practice Rules:
% ------------------------------------
% * Use \cref{<label>} only, no \ref !
% * For abbreviations, always use the Glossaries package and \gls{<abbrv>} or
%   \glspl{<abbrv>} ! --> ensures that every abbreviation is introduced correctly
% * Mark open points with \todo{ <Name of assigned person>: ... }
% * Only commit after checking that it compiles. 
% * Only one sentence per line!
% * Use active voice whenever possible! ("we have investigated" instead of "it
%   was investigated")
% * Use past progressive instead of simple past! ("Banks et al. have shown"
%   instead of "Banks et al. showed")

%% glossaries package for list of abbreviations and list of symbols
%\usepackage{datatool} % package needed by glossaries
%\usepackage[acronym]{glossaries} % *after* hyperref
%\newglossary[symlog]{symbol}{symi}{symo}{Symbols}
%\makeglossaries
%\glstoctrue
%\input{../../paper.library/abbreviations_symbols/abbreviations}
%\input{../../paper.library/abbreviations_symbols/symbols_finance}


%%%%%%%%%%%%%%%%%%%%%%%%%%%%%%%%%%%%%%%%%%%%%%%%%%%%%%%%%%%%%%%%%%%%%%%
%% EIT Package and Language Settings
%%%%%%%%%%%%%%%%%%%%%%%%%%%%%%%%%%%%%%%%%%%%%%%%%%%%%%%%%%%%%%%%%%%%%%

% for English documents (standard)
%\usepackage{style-eit-latex/EIT}

% use [de] option for German documents
\usepackage[de]{style-eit-latex/EIT}


%%%%%%%%%%%%%%%%%%%%%%%%%%%%%%%%%%%%%%%%%%%%%%%%%%%%%%%%%%%%%%%%%%%%%%%
%% Headers & Footers
%%%%%%%%%%%%%%%%%%%%%%%%%%%%%%%%%%%%%%%%%%%%%%%%%%%%%%%%%%%%%%%%%%%%%%

% Header info here:
%\ihead
%	{\color{tuk-grey} \pt
%		<head left>
%	}
%\chead
%	{\color{tuk-grey} \pt
%		<head center>
%	}
%\ohead
%	{\color{tuk-grey} \pt
%		<head right>
%	}
%% Footer info here:
\ifoot
	% Footer left first page
	[\color{tuk-grey} \pt
		GAWT U05b
	]
%	% Footer left other pages
	{\color{tuk-grey} \pt
		GAWT U05b
	}
%\cfoot
%	% Footer center first page
%	[\color{tuk-grey} \pt
%		<footer center first page>
%	]
%	% Footer center other pages
%	{\color{tuk-grey} \pt
%		<footer center other pages>
%	}
%\ofoot
%	% Footer right first page
%	[\color{tuk-grey} \pt
%		\iftoggle{isGerman}{
%  			Seite \thepage\ von \pageref{LastPage}
%		}{
%			Page \thepage\ of \pageref{LastPage}
%		}
%	]
%	% Footer right other pages
%	{\color{tuk-grey} \pt
%		\iftoggle{isGerman}{
%  			Seite \thepage\ von \pageref{LastPage}
%		}{
%			Page \thepage\ of \pageref{LastPage}
%		}
%	}

%%%%%%%%%%%%%%%%%%%%%%%%%%%%%%%%%%%%%%%%%%%%%%%%%%%%%%%%%%%%%%%%%%%%%%%
%% Hyphenation
%%%%%%%%%%%%%%%%%%%%%%%%%%%%%%%%%%%%%%%%%%%%%%%%%%%%%%%%%%%%%%%%%%%%%%

% correct bad hyphenation here
\hyphenation{net-works semi-conduc-tor quad-ra-ture mat-ches trans-form meth-od}


%%%%%%%%%%%%%%%%%%%%%%%%%%%%%%%%%%%%%%%%%%%%%%%%%%%%%%%%%%%%%%%%%%%%%%%
%% Glossary
%%%%%%%%%%%%%%%%%%%%%%%%%%%%%%%%%%%%%%%%%%%%%%%%%%%%%%%%%%%%%%%%%%%%%%

% enable glossary generation
%\makeglossaries


%%%%%%%%%%%%%%%%%%%%%%%%%%%%%%%%%%%%%%%%%%%%%%%%%%%%%%%%%%%%%%%%%%%%%%%
%% Title and Document Information
%%%%%%%%%%%%%%%%%%%%%%%%%%%%%%%%%%%%%%%%%%%%%%%%%%%%%%%%%%%%%%%%%%%%%%

\begin{document}

% Title, author and other information:
\title{<Your Title>}
\subtitle{<Sub-Title>}
\author{<Your Name>}

%\maketitle


%%%%%%%%%%%%%%%%%%%%%%%%%%%%%%%%%%%%%%%%%%%%%%%%%%%%%%%%%%%%%%%%%%%%%%%
%% Font Selection
%%%%%%%%%%%%%%%%%%%%%%%%%%%%%%%%%%%%%%%%%%%%%%%%%%%%%%%%%%%%%%%%%%%%%%

% If you want to have serif font please comment the following command: 
\pt 


%%%%%%%%%%%%%%%%%%%%%%%%%%%%%%%%%%%%%%%%%%%%%%%%%%%%%%%%%%%%%%%%%%%%%%%
%% Body Text
%%%%%%%%%%%%%%%%%%%%%%%%%%%%%%%%%%%%%%%%%%%%%%%%%%%%%%%%%%%%%%%%%%%%%%


\section*{Verteilungsfunktion der Standardnormalverteilung}
% from https://www.york.ac.uk/depts/maths/tables/normal.htm
\vspace{-1cm}
\begin{center}
	$$F_x(x) = \int_{-\infty}^{x} \frac{1}{\sqrt{2 \pi}} \cdot e^{-\frac{\xi^2}{2}} d\xi$$
	\small
	\begin{tabular}{rr@{\ }r@{\ }r@{\ }r@{\ }r@{\ }r@{\ }r@{\ }r@{\ }r@{\ }r@{\ }r}
		$x$&0.00&0.01&0.02&0.03&0.04&0.05&0.06&0.07&0.08&0.09\\
		\ \\
		0.0&0.5000&0.5040&0.5080&0.5120&0.5160&0.5199&0.5239&0.5279&0.5319&0.5359\\
		0.1&0.5398&0.5438&0.5478&0.5517&0.5557&0.5596&0.5636&0.5675&0.5714&0.5753\\
		0.2&0.5793&0.5832&0.5871&0.5910&0.5948&0.5987&0.6026&0.6064&0.6103&0.6141\\
		0.3&0.6179&0.6217&0.6255&0.6293&0.6331&0.6368&0.6406&0.6443&0.6480&0.6517\\
		0.4&0.6554&0.6591&0.6628&0.6664&0.6700&0.6736&0.6772&0.6808&0.6844&0.6879\\
		0.5&0.6915&0.6950&0.6985&0.7019&0.7054&0.7088&0.7123&0.7157&0.7190&0.7224\\
		0.6&0.7257&0.7291&0.7324&0.7357&0.7389&0.7422&0.7454&0.7486&0.7517&0.7549\\
		0.7&0.7580&0.7611&0.7642&0.7673&0.7703&0.7734&0.7764&0.7794&0.7823&0.7852\\
		0.8&0.7881&0.7910&0.7939&0.7967&0.7995&0.8023&0.8051&0.8078&0.8106&0.8133\\
		0.9&0.8159&0.8186&0.8212&0.8238&0.8264&0.8289&0.8315&0.8340&0.8365&0.8389\\
		1.0&0.8413&0.8438&0.8461&0.8485&0.8508&0.8531&0.8554&0.8577&0.8599&0.8621\\
		1.1&0.8643&0.8665&0.8686&0.8708&0.8729&0.8749&0.8770&0.8790&0.8810&0.8830\\
		1.2&0.8849&0.8869&0.8888&0.8907&0.8925&0.8944&0.8962&0.8980&0.8997&0.9015\\
		1.3&0.9032&0.9049&0.9066&0.9082&0.9099&0.9115&0.9131&0.9147&0.9162&0.9177\\
		1.4&0.9192&0.9207&0.9222&0.9236&0.9251&0.9265&0.9279&0.9292&0.9306&0.9319\\
		1.5&0.9332&0.9345&0.9357&0.9370&0.9382&0.9394&0.9406&0.9418&0.9429&0.9441\\
		1.6&0.9452&0.9463&0.9474&0.9484&0.9495&0.9505&0.9515&0.9525&0.9535&0.9545\\
		1.7&0.9554&0.9564&0.9573&0.9582&0.9591&0.9599&0.9608&0.9616&0.9625&0.9633\\
		1.8&0.9641&0.9649&0.9656&0.9664&0.9671&0.9678&0.9686&0.9693&0.9699&0.9706\\
		1.9&0.9713&0.9719&0.9726&0.9732&0.9738&0.9744&0.9750&0.9756&0.9761&0.9767\\
		2.0&0.9772&0.9778&0.9783&0.9788&0.9793&0.9798&0.9803&0.9808&0.9812&0.9817\\
		2.1&0.9821&0.9826&0.9830&0.9834&0.9838&0.9842&0.9846&0.9850&0.9854&0.9857\\
		2.2&0.9861&0.9864&0.9868&0.9871&0.9875&0.9878&0.9881&0.9884&0.9887&0.9890\\
		2.3&0.9893&0.9896&0.9898&0.9901&0.9904&0.9906&0.9909&0.9911&0.9913&0.9916\\
		2.4&0.9918&0.9920&0.9922&0.9925&0.9927&0.9929&0.9931&0.9932&0.9934&0.9936\\
		2.5&0.9938&0.9940&0.9941&0.9943&0.9945&0.9946&0.9948&0.9949&0.9951&0.9952\\
		2.6&0.9953&0.9955&0.9956&0.9957&0.9959&0.9960&0.9961&0.9962&0.9963&0.9964\\
		2.7&0.9965&0.9966&0.9967&0.9968&0.9969&0.9970&0.9971&0.9972&0.9973&0.9974\\
		2.8&0.9974&0.9975&0.9976&0.9977&0.9977&0.9978&0.9979&0.9979&0.9980&0.9981\\
		2.9&0.9981&0.9982&0.9982&0.9983&0.9984&0.9984&0.9985&0.9985&0.9986&0.9986\\
		3.0&0.9987&0.9987&0.9987&0.9988&0.9988&0.9989&0.9989&0.9989&0.9990&0.9990\\
		3.1&0.9990&0.9991&0.9991&0.9991&0.9992&0.9992&0.9992&0.9992&0.9993&0.9993\\
		3.2&0.9993&0.9993&0.9994&0.9994&0.9994&0.9994&0.9994&0.9995&0.9995&0.9995\\
		3.3&0.9995&0.9995&0.9995&0.9996&0.9996&0.9996&0.9996&0.9996&0.9996&0.9997\\
		3.4&0.9997&0.9997&0.9997&0.9997&0.9997&0.9997&0.9997&0.9997&0.9997&0.9998\\
		3.5&0.9998&0.9998&0.9998&0.9998&0.9998&0.9998&0.9998&0.9998&0.9998&0.9998\\
		3.6&0.9998&0.9998&0.9999&0.9999&0.9999&0.9999&0.9999&0.9999&0.9999&0.9999\\
		3.7&0.9999&0.9999&0.9999&0.9999&0.9999&0.9999&0.9999&0.9999&0.9999&0.9999\\
		3.8&0.9999&0.9999&0.9999&0.9999&0.9999&0.9999&0.9999&0.9999&0.9999&0.9999\\
		3.9&1.0000&1.0000&1.0000&1.0000&1.0000&1.0000&1.0000&1.0000&1.0000&1.0000\\
	\end{tabular}
\end{center}

%\bibliographystyle{plain}
%\bibliography{<path to your bibliography>}

\end{document}
